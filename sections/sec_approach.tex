\section{Approach}
\label{sec:Approach}

%______________________________________________________________________

\subsection{Information Selection}
\label{subsec:InformationSelection}

%______________________________________________________________________

\subsubsection{Selection Process}
\label{subsubsec:SelectionProcess}
The information selection was done in online libraries. Some of those online libraries are open-access, others require authentication. 
INSA Lyon offers access from the \textit{Biblioth\`{e}que Marie Curie}\footnote{\url{http://scd.docinsa.insa-lyon.fr/voir_tout2p.php}}. The following online databases were used:
\begin{itemize}[noitemsep]
	\item Emerald Insight
	\item Ebooks on Ebsco
	\item Springer
	\item Thèses INSA
\end{itemize}
University of Passau offers access from the \textit{Universit\"{a}tsbibliothek Passau (DBIS)}\footnote{\url{http://dbis.uni-regensburg.de/fachliste.php?bib_id=ub_pa&lett=l&colors=&ocolors=}}. The following online databases were used:
\begin{itemize}[noitemsep]
	\item ACM Digital Library
	\item IEEE Explore
	\item arXiv.org
	\item De Gruyter Online / E-Books
	\item Springer eBooks: Computer Science
	\item Springer eBooks: Technik und Informatik
\end{itemize}
Also, some free online libraries were used during the research:
\begin{itemize}[noitemsep]
	\item Google Scholar
	\item Google Books
	\item Google Search Engine
\end{itemize}
The main key words that were used during the search were: Blockchain, Bitcoin, Distributed Technology, Distributed Ledger Technology, Distributed Networks, Hyperledger, Use Case. The terms here were inspired by the keywords used in the scoping review conducted by \cite{2018_Li}.


%______________________________________________________________________

\begin{landscape}
\subsubsection{Selected Papers}
\label{subsubsec:selected_papers}
\begin{center}
\begin{tabular}{ |c|c|c| }
	\hline
 	Publication Channel & Rating &  Paper(s) \\ [0.5ex] 
 	\hline\hline
	 ACM Conference on Digital Libraries (JCDL) & A* & \citet{2017_Gipp} \\ 
	 \hline
	 IEEE International Conference on Computer Communications (IEEE INFOCOM) & A* & \citet{2016_Kianmajd} \\ 
	 \hline
	  International Conference on Information Systems (ICIS) & A* & \citet{2017_Naerland} \\ 
	 \hline
	  IEEE Symposium on Security and Privacy (S\&P) & A* & \citet{2015_Zyskind} \\ 
	 \hline
 		Americas Conference on Information Systems (AMCIS) & A & \citet{2017_Madhwal} \\ 
	 \hline
	 European Symposium On Research In Computer Security (ESORICS) & A & \citet{2017_Tackmann} \\ 
	 \hline
	 IFIP Information Security \& Privacy Conference (IFIP SEC) & B & \citet{2016_Schaub} \\
	 \hline
	 International Computer Software and Applications Conference (COMPSAC) & B & \citet{2016_Yasin} \\ 
	 \hline
	 Journal of Medical Systems & C & \citet{2016_Yue} \\ 
	 \hline
	 Journal of Software Engineering and Applications & Not ranked & \citet{2016_Bahga} \\ 
	 \hline
	 International Journal on Advanced Science, Engineering and Information Technology & N\textbackslash A & \citet{2018_Alessandra} \\ 
	 \hline
	 International Conference on Open and Big Data (OBD) & N\textbackslash A & \citet{2016_Azaria} \\ 
	 \hline
	 Journal of Emerging Technologies in Accounting & N\textbackslash A & \citet{2017_Coyne} \\ 
	 \hline
	 International Conference for Internet Technology and Secured Transactions (ICITST) & N\textbackslash A & \citet{2015_Dennis} \\ 
	 \hline
	 Book \enquote{The Changing Postal and Delivery Sector} & N\textbackslash A & \citet{2017_Jaag} \\ 
	 \hline
	 International Conference on Distributed Computing and Artificial Intelligence (DCAI) & N\textbackslash A & \citet{2016_Jacynycz} \\ 
	 \hline
	 N\textbackslash A & N\textbackslash A & \citet{2017_Liu} \\ 
	 \hline
	Symposium on Foundations and Applications of Blockchain (FAB) & N\textbackslash A & \citet{2018_Lucena} \\ 
	 \hline
	 Europe and MENA Cooperation Advances in Information and Communication Technologies & N\textbackslash A & \citet{2017_Ouaddah} \\ 
	 \hline
	 European Conference on Technology Enhanced Learning (EC-TEL) & N\textbackslash A & \citet{2016_Sharples} \\ 
	 \hline
	 International Conference on Service Systems and Service Management (ICSSSM) & N\textbackslash A & \citet{2016_Tian} \\ 
	 \hline
\end{tabular}
\end{center}

\end{landscape}

%______________________________________________________________________

\clearpage
\subsection{My schema / Research questions}
While doing research on for this ‘Blockchain’ review, everyone I talked to kept on asking me ‘WHY?’ would you use this technology. Since I was fairly new to the topic, I could not really respond. While reading, I understood the technical basics of the technology and how \cite{2009_Nakamoto} solved the double-spending problem that occurred in the area of cryptocurrency. I afterwards looked for more cases that the technology was used for and found a lot of experiments. This got me curious and I wanted a review that told me the reasons, why the researches experimented with the Blockchain. Unfortunately, in all the review papers that I read, see chapter \ref{subsec:BlockchainReviews}, there was no one that focused on this question and most important, gave a good overview about this. I then asked myself what I wanted to know about the use cases and about the technology Blockchain. I took a look at the basic questions used to gather information: \cite{2002_Hart}
\begin{itemize}[noitemsep]
	\item \textbf{Who} was involved?
	\item \textbf{What} happened?
	\item \textbf{Where} did it take place?
	\item \textbf{When} did it take place?
	\item \textbf{Why} did it that happen?
	\item \textbf{How} did it happen? 
\end{itemize}
I then established that there were two different categories that I wanted to analyze: the general and the content-related information about the paper.


%______________________________________________________________________

\subsubsection{General Information}
\label{subsubsec:general_information}
In their publication, \citet{2018_Li} used the following properties as general information: publication year, geographic distribution, publication type and nature of studies. Combining this with the 6 questions of the beginning, I established my own questions and formulated my research questions (RQ)\nomenclature[O]{RQ(s)}{Research Question(s)}:
\begin{itemize}[noitemsep]
	\item \textbf{RQ1}) Publisher: \textbf{Who} published the paper?
	\item \textbf{RQ2}) Publication Type: \textbf{What} was the publication type of the paper?
	\item \textbf{RQ3}) Publication Channel: \textbf{Where} was the paper published?
	\item \textbf{RQ4}) Publication Date: \textbf{When} was the paper published?
\end{itemize}
For the general information, the questions \quoteit{Why} and \quoteit{How} were not relevant. Instead the keywords that were selected by the authors were put into consideration:
\begin{itemize}
	\item \textbf{RQ5}) Keywords: \textbf{\enquote{Why} \enquote{How}}: What kind of keywords were used?
\end{itemize}

%______________________________________________________________________

\subsubsection{Content-based Information}
For the second category, the content-related information, the publication of \citet{2018_Li} was considered again, since he defined to following schema:
\begin{itemize}[noitemsep]
	\item \textbf{What} is the nature of the used blockchain application?
	\item \textbf{Why} did they invest in the blockchain technology?
	\item \textbf{Whom} did they target with the blockchain technology?
	\item \textbf{How} does their implementation of the blockchain work and operate?
\end{itemize}
Combinig his ideas to my 6 questions again and rephrasing them, so they answer my primary questions. The questions \quoteit{Where} and \quoteit{When} were not treated in this section because they were not relevant. The question \quoteit{Conclusion} was added in the end, to provide an overview of the outcome of the various use cases.
My research questions in this category are:
\begin{itemize}[noitemsep]
	\item \textbf{RQ6}) Authors: \textbf{Who} conducted the research?
	\item \textbf{RQ7}) Contribution: \textbf{What} did they analyze? What is their contribution?
	\item \textbf{RQ8}) Reason/Problems: \textbf{Why} did they analyze this(these) problem(s)? And why do they think Blockchain could solve their problem(s)?
	\item \textbf{RQ9}) Implementation Process: \textbf{How?} What blockchain technology did they use and how did they implement it?
	\item \textbf{RQ10}) Conclusion: What was their conclusion? Does is make sense to use the Blockchain to solve the problem they selected? Any benefits or drawbacks?
\end{itemize}