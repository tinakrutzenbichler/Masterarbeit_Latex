\section*{Summary Review articles}
\subsection*{\citet{2018_Li}}
The review article was published by \citet{2018_Li} in 2018. Its main goal is to determine the size and scope of the business literature on the blockchain technology. During their scan, they found 39 papers that fit all their criteria.\\ In the second chapter (pp. 4475 - 4477), the authors describe their technique for reviewing the existing literature. As described earlier, I oriented my search also based on their keywords, but I used different online libraries. They created a 3x4 matrix to classify their information. It refers on the one hand to the focus of the paper. Here they established 4 different sections: 
\begin{itemize}[noitemsep]
	\item What? What kind of blockchain application was used in the publication?
	\item Why? Why did the authors decide to use blockchain technology?
	\item Whom? Who did they implement the blockchain for?
	\item How? How did they implement the blockchain?
\end{itemize}
And on the other hand, it shows the level of application and is divided into individuals, firms and governments.
The third chapter (pp. 4477 - 4480) covers the results that were found. Here, they divided into four sections:
\begin{description}[align=left]
	\item [Publication year and geographic distribution] In this section, they name the publication year of every paper. Also, they took a look at where, geographically, the paper was published (the research was conducted).
	\item [Publication type and nature of studies] This part gives an overview of the different types of papers that were used and the method of scientific research that was conducted. Most papers were conference proceedings and the main method used was conceptual research.
	\item [Main applications of blockchain in business organizations] 76.9 \% of the selected 39 papers discuss the different usages of the blockchain technology in business organizations. The rest 23.1\% concentrate on other subjects, for example opportunities, risks or challenges.
	\item [Focus of inquiry and level of analysis] Classification of the papers according to their framework that they developed for their research. X axis and y axis. Most of the papers focused on the how for firms.
\end{description} 
Their analysis was described in the fourth chapter (pp. 4480 - 4481). The authors found a few gaps in the current literature on the subject:
\begin{itemize}
	\item Not a lot of empirical studies have been conducted and there is a big absence of these studies that examine why companies would invest and adopt the blockchain.
	\item In their classification, they had the whom question, which analyzed the impact of blockchain on different actors. They think that in the literature this is not documented yet.
\end{itemize}
Nevertheless, they also say that they had some limitations during their search. The first one was the language restriction to English and also their search strategy. Secondly, the risk of selection bias was there. They concluded by saying that in the end there was no panel of experts that validated their findings, like it would normally be done. \\
The scoping review that was done in this paper was well conducted and scientifically very well described. Also, as mentioned on (p. 4479), the full list of blockchain usages, that they used, is available upon request (Types of Blockchain Usage, 2018). Therefore, I mailed the leading author and he send me a list of all used articles within a few days. \\
Reading this review, I expected to find more technical aspects, but they were missing completely. I got a very good selection of papers, but I could not say which one is good or which blockchain implementation I would use, because they never presented an overview in the end. Also, the list the author mailed includes 22 articles that are types of blockchain usage. Yet it was a little hard to find the articles because he did not put the original title of the respective publication. In the end, I was quite disappointed to not get an overview of all article and criteria to be able to see the difference between the articles they selected.\\
Overall a very good scientific publication which provides a good selection of literature of blockchain use cases in the business area.\\ 
\subsection*{\citet{2016_Karame}}
\cite{2016_Karame}\\
Objective:\\
In their work, they present a tutorial that overviews, details and analyzes the security measurements done by Bitcoin and the underlying Blockchain.\\
Contribution:\\
captured recent reported attacks or threats in the systems
He mentions three challenges that Bitcoin has to overcome at the time:
Security - \cite{2015_Gervais}, \cite{2015_Heilman} \cite{2012_Karame}
Scalability 
Limits of (De-)centralization \cite{2014_Gervais}